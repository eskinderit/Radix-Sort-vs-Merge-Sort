\documentclass[a4paper]{article}
\usepackage[italian]{babel}
\font\TitleFont=cmr12 at 50pt
\font\AuthFont=cmr13 at 20pt
\font\ChapFont=cmr12 at 30 pt
\usepackage{titling}
\usepackage{graphicx}
\usepackage[section]{placeins}
\usepackage{tabularx}

\title {{\TitleFont Esercizio 3}}
\date{3 Luglio 2019}
\author{{\AuthFont Alessandro D'Amico}}


\renewcommand\maketitlehooka{\null\mbox{}\vfill}
\renewcommand\maketitlehookd{\vfill\null}

\begin{document}
\begin{titlingpage}
\maketitle
\end{titlingpage}
\tableofcontents
\newpage
\section{Introduzione al problema}
Nel seguente esperimento viene preso in considerazione il problema dell'ordinamento, per la cui soluzione utilizziamo l'algoritmo Radix Sort.
Lo scopo e' ottenere un array (una lista di numeri) i cui elementi che lo compongono (inizialmente disposti in modo casuale) siano disposti in ordine crescente. I dataset utilizzati contengono numeri binari corrispondenti ad interi positivi.
\section{Caratteristiche teoriche di algoritmi e e strutture utilizzate}
A differenza di Merge Sort, Radix Sort non e' un algoritmo per confronto e pertanto non effettua comparazioni per stabilire una relazione di $\leq$
\section{Prestazioni attese}
	\begin{tabularx}{10cm}{|X|X|}
	\hline
	Algoritmo & Tempo di esecuzione caso peggiore \\
	\hline
	\textbf{Radix Sort} &  $\Theta(d(k+n))$ \\
	\hline
	\textbf{Merge Sort} &  $\Theta(nlg(n))$ \\
	\hline
	\end{tabularx}
\section{Esperimenti}
\section{Documentazione del codice}
\newpage
\section{Risultati}
\subsection{Radix Sort al variare del passo}
\subsection{Radix Sort Vs Merge Sort}

\section{Conclusioni}

\end{document}
